\documentclass{article}
\usepackage{amsmath, amssymb}
\begin{document}

\section*{Six-Basis Representation and Conversions}

Let \(\phi = \frac{1+\sqrt{5}}{2}\) denote the golden ratio. We define six basis vectors in \(\mathbb{R}^3\) associated with the dodecahedron by
\[
\begin{array}{rcl}
\mathbf{B}_1 &=& (0,\, 1,\, \phi),\\[1mm]
\mathbf{B}_2 &=& (0,\, 1,\,-\phi),\\[1mm]
\mathbf{B}_3 &=& (1,\, \phi,\, 0),\\[1mm]
\mathbf{B}_4 &=& (1,\,-\phi,\, 0),\\[1mm]
\mathbf{B}_5 &=& (\phi,\, 0,\, 1),\\[1mm]
\mathbf{B}_6 &=& (-\phi,\, 0,\, 1).
\end{array}
\]

Any point in our structure is uniquely represented by a \emph{legal} six–vector 
\[
\mathbf{v} = [a,\, b,\, c,\, d,\, e,\, f] \in \mathbb{Z}^6,
\]
with the inherent consistency that only three of these six integers are independent. In our framework the six–vector serves as an address in a crystalline lattice, and all arithmetic remains entirely in this space.

\subsection*{Conversion to Cartesian Coordinates}

A point \(\mathbf{v} = [a,\, b,\, c,\, d,\, e,\, f]\) is mapped to Cartesian coordinates \((x,y,z) \in \mathbb{R}^3\) by the linear transformation
\[
\begin{aligned}
x &= c + d + \phi\,(e + f),\\[1mm]
y &= a + b + \phi\,(c - d),\\[1mm]
z &= \phi\,(a - b) + (e - f).
\end{aligned}
\]
Thus, the mapping is expressed as
\[
\mathbf{v} \mapsto (x,y,z) = \bigl( c+d+\phi(e+f),\; a+b+\phi(c-d),\; \phi(a-b)+(e-f) \bigr).
\]

\subsection*{Conversion from Cartesian Coordinates}

Since every Cartesian coordinate lies in the quadratic integer ring \(\mathbb{Z}[\phi]\), we can uniquely write
\[
x = X_0 + \phi X_1,\quad y = Y_0 + \phi Y_1,\quad z = Z_0 + \phi Z_1,
\]
with \(X_0,X_1,Y_0,Y_1,Z_0,Z_1 \in \mathbb{Z}\). Matching coefficients with the above transformation, we have:
\[
\begin{array}{rcl}
c+d &=& X_0,\quad  e+f = X_1,\\[1mm]
a+b &=& Y_0,\quad  c-d = Y_1,\\[1mm]
\phi(a-b) + (e-f) &=& Z_0 + \phi Z_1.
\end{array}
\]
A convenient identification is to set
\[
e-f = Z_0 \quad \text{and} \quad a-b = Z_1.
\]
Then, the inversion formulas become
\[
\begin{aligned}
a &= \frac{Y_0 + Z_1}{2}, \quad &b &= \frac{Y_0 - Z_1}{2},\\[1mm]
c &= \frac{X_0 + Y_1}{2}, \quad &d &= \frac{X_0 - Y_1}{2},\\[1mm]
e &= \frac{X_1 + Z_0}{2}, \quad &f &= \frac{X_1 - Z_0}{2}.
\end{aligned}
\]
This provides the unique legal six–vector corresponding to the Cartesian point \((x,y,z)\), when \(x\), \(y\), and \(z\) are expressed in their \(\mathbb{Z}[\phi]\) decompositions.

\subsection*{Summary}

Every legal point in our crystalline lattice is represented by a six–vector 
\[
\mathbf{v} = [a,\, b,\, c,\, d,\, e,\, f] \in \mathbb{Z}^6,
\]
with the mapping to Cartesian space given by
\[
\begin{pmatrix} x \\ y \\ z \end{pmatrix}
=
\begin{pmatrix}
c+d+\phi(e+f)\\[1mm]
a+b+\phi(c-d)\\[1mm]
\phi(a-b)+(e-f)
\end{pmatrix},
\]
and the inverse mapping (when Cartesian coordinates are written as \(x=X_0+\phi X_1\), etc.) by
\[
\begin{aligned}
a &= \frac{Y_0 + Z_1}{2}, \quad b = \frac{Y_0 - Z_1}{2},\\[1mm]
c &= \frac{X_0 + Y_1}{2}, \quad d = \frac{X_0 - Y_1}{2},\\[1mm]
e &= \frac{X_1 + Z_0}{2}, \quad f = \frac{X_1 - Z_0}{2}.
\end{aligned}
\]

This fixed framework is used throughout our construction of structures from Golden Triangles (T) and Golden Gnomons (G), ensuring that all placements and legal displacements are managed entirely within the six–space coordinate system.

\end{document}
